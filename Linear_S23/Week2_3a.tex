\documentclass[reqno]{amsart}


\pagestyle{empty}

\usepackage{graphicx}
\usepackage[margin = 1cm]{geometry}
\usepackage{color}
\usepackage{cancel}
\usepackage{multirow}
\usepackage{framed}

\newtheorem{thm}{Theorem}
\newtheorem{cor}{Corollary}
\theoremstyle{definition}
\newtheorem{definition}{Definition}

\begin{document}
\begin{flushleft}
{\sc \Large AMATH 352 Rahman} \hfill Week 2
\bigskip
\end{flushleft}

\newcommand{\R}{\mathbb{R}}
\newcommand{\N}{\mathbb{N}}
\newcommand{\Z}{\mathbb{Z}}
\newcommand{\Q}{\mathbb{Q}}
\renewcommand{\CancelColor}{\color{red}}
\newcommand{\?}{\stackrel{?}{=}}
\renewcommand{\varphi}{\phi}
\newcommand{\card}{\text{Card}}
\newcommand{\bigzero}{\text{\Huge 0}}



\section*{Sec. 1.3 Gaussian Elimination}

Now lets go back and do a bunch of Gaussian Elimination problems.  Again consider our equation from last time
%
\begin{equation}
\begin{split}
2u + v + w &= 5\\
4u - 6v &= -2\\
-2u + 7v + 2w &= 9
\end{split}
\end{equation}
%
we will write this as an augmented matrix by appending the right hand side (RHS) to the coefficient matrix,
%
\begin{equation*}
\begin{matrix}
\\
{\color{red} 2}\\
\\
\end{matrix}\begin{bmatrix}
2 & 1 & 1 & | & {\color{blue}5}\\
4 & -6 & 0 & | & {\color{blue}-2}\\
-2 & 7 & 2 & | & {\color{blue}9}
\end{bmatrix} = \begin{matrix}
\\
\\
{\color{red}-1}
\end{matrix}\begin{bmatrix}
2 & 1 & 1 & | & {\color{blue}5}\\
0 & -8 & -2 & | & {\color{blue}-12}\\
-2 & 7 & 2 & | & {\color{blue}9}
\end{bmatrix} = \begin{matrix}
\\
\\
{\color{red}-1}
\end{matrix} \begin{bmatrix}
2 & 1 & 1 & | & {\color{blue}5}\\
0 & -8 & -2 & | & {\color{blue}-12}\\
0 & 8 & 3 & | & {\color{blue}14}
\end{bmatrix} = \begin{bmatrix}
2 & 1 & 1 & | & {\color{blue}5}\\
0 & -8 & -2 & | & {\color{blue}-12}\\
0 & 0 & 1 & | & {\color{blue}2}
\end{bmatrix}
\end{equation*}
%
This means $\fbox{w = 2}$, then we plug into the second equation to get $\fbox{v = 1}$,
and finally the first to get $\fbox{u = 1}$.

The elements down the diagonal are called \underline{pivots}.  The augmented matrix is said to be in 
\underline{row-echelon} form.  The original matrix,
%
\begin{equation*}
\begin{bmatrix}
2 & 1 & 1\\
0 & -8 & -2\\
0 & 0 & 1
\end{bmatrix}
\end{equation*}
%
is said to be in \underline{upper triangular form}.


Here are a few more Gaussian elimination examples.
%
\begin{enumerate}

\item[Ex:   ]  

\begin{equation*}
\begin{matrix}
\\
{\color{red}3}
\end{matrix}\begin{bmatrix}
1 & 3 & | & {\color{blue}11}\\
3 & 1 & | & {\color{blue}9}
\end{bmatrix} = \begin{bmatrix}
1 & 3 & | & {\color{blue}11}\\
0 & -8 & | & {\color{blue}-24}
\end{bmatrix}
\Rightarrow \fbox{y = 3} \Rightarrow \fbox{x = 2};
\end{equation*}

\item[Ex:   ]  

\begin{equation*}
\begin{matrix}
\\
-2
\end{matrix}
\begin{bmatrix}
-1 & 2 & | & {\color{blue}3/2}\\
2 & -4 & | & {\color{blue}3}
\end{bmatrix} = \begin{bmatrix}
-1 & 2 & | & {\color{blue}3/2}\\
0 & 0 & | & {\color{blue}-6}
\end{bmatrix}
\end{equation*}

Clearly this matrix is singular, and since the RHS is nontrivial it will have \fbox{no solutions}.

\item[Ex:   ]  

\begin{equation*}
\begin{matrix}
\\
{\color{red}3}\\
{\color{red}2}
\end{matrix}
\begin{bmatrix}
1 & 0 & -3 & | & {\color{blue}-2}\\
3 & 1 & -2 & | & {\color{blue}5}\\
2 & 2 & 1 & | & {\color{blue}4}
\end{bmatrix} = \begin{matrix}
\\
\\
{\color{red}2}
\end{matrix}\begin{bmatrix}
1 & 0 & -3 & | & {\color{blue}-2}\\
0 & 1 & 7 & | & {\color{blue}11}\\
0 & 2 & 7 & | & {\color{blue}8}
\end{bmatrix} = \begin{bmatrix}
1 & 0 & -3 & | & {\color{blue}-2}\\
0 & 1 & 7 & | & {\color{blue}11}\\
0 & 0 & -7 & | & {\color{blue}-14}
\end{bmatrix}
\end{equation*}
%
then $x_3 = 2$, $x_2 = -3$, $x_1 = 4$.

\item[Ex:   ]  

\begin{equation*}
\begin{matrix}
\\
{\color{red}2}\\
{\color{red}4}
\end{matrix}
\begin{bmatrix}
2 & 0 & 3 & | & {\color{blue}3}\\
4 & -3 & 7 & | & {\color{blue}5}\\
8 & -9 & 15 & | & {\color{blue}10}
\end{bmatrix} = \begin{matrix}
\\
\\
{\color{red}3}
\end{matrix}\begin{bmatrix}
2 & 0 & 3 & | & {\color{blue}3}\\
0 & -3 & 1 & | &  {\color{blue}-1}\\
0 & -9 & 3 & | &  {\color{blue}-2}
\end{bmatrix} = \begin{bmatrix}
2 & 0 & 3 & | & {\color{blue}3}\\
0 & -3 & 1 & | &  {\color{blue}-1}\\
0 & 0 & 0 & | & {\color{blue}1}
\end{bmatrix}
\end{equation*}

Clearly this matrix is singular, and since the RHS is nontrivial it will have \fbox{no solutions}.

\end{enumerate}

\end{document}