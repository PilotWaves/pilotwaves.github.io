\documentclass[reqno]{amsart}


\pagestyle{empty}

\usepackage{graphicx}
\usepackage[margin = 1cm]{geometry}
\usepackage{color}
\usepackage{cancel}
\usepackage{multirow}
\usepackage{framed}
\usepackage{amssymb}
\usepackage{stackengine}
\usepackage{tikz}

\newtheorem{thm}{Theorem}
\newtheorem{cor}{Corollary}
\theoremstyle{definition}
\newtheorem{definition}{Definition}

\newenvironment{handwave}{%
  \renewcommand{\proofname}{Handwavey proof}\proof}{\endproof}
  %\renewcommand{\qedsymbol}{$\blacksquare$}

\begin{document}
\begin{flushleft}
{\sc \Large AMATH 352 Rahman} \hfill Week 7
\bigskip
\end{flushleft}

\newcommand{\R}{\mathbb{R}}
\newcommand{\N}{\mathbb{N}}
\newcommand{\Z}{\mathbb{Z}}
\newcommand{\Q}{\mathbb{Q}}
\renewcommand{\CancelColor}{\color{red}}
\newcommand{\?}{\stackrel{?}{=}}
\renewcommand{\varphi}{\phi}
\newcommand{\card}{\text{Card}}
\newcommand{\bigzero}{\text{\Huge 0}}
\newcommand{\curvearrowdown}{{\color{red}\rotatebox{90}{$\curvearrowleft$}}}
\newcommand{\curvearrowup}{{\color{red}\rotatebox{90}{$\curvearrowright$}}}

\newcommand*\circled[1]{\color{red}\tikz[baseline=(char.base)]{
            \node[shape=circle,draw,inner sep=2pt] (char) {#1};}}



\section*{Sec. 3.1 and 4.1 Inner products and orthogonality}

We briefly reviewed how to do dot products.  Then we talked about orthogonality and projections.

\bigskip

\underline{Brief dot product review}

{\color{red}You would have seen this in previous courses, especially Calc III}

\begin{itemize}

\item  Length of a vector: $\|v\| = \sqrt{v_1^2 + v_2^2 + \cdots + v_n^2} = \sqrt{v\cdot v}$.
\item  A unit vector is a vector that has length one.  Any vector $v$ can be turned into a unit vector by dividing by its length: $v/\|v\|$.  This type of vector is also said to be \underline{normal}.
\item  Distance between vectors: $\|u-v\| = \sqrt{(u_1-v_1)^2 + (u_2-v_2)^2}$.
\item  Angle between vectors (law of cosines): $\cos\theta = u\cdot v/\|u\|\|v\|$.  Notice that a right angle $\theta = \pi/2$ implies $u\cdot v = 0$, so this is another way that we can show two vectors are perpendicular.
\item  Pythagorean theorem:  $\|u+v\|^2 = \|u\|^2 + \|v\|^2$.

\end{itemize}

\begin{definition}
Consider $\vec{u},\vec{v},\vec{w} \in V \subseteq \R^n$.  The product $\langle \vec{v},\vec{w}\rangle \in \R$ is said to be an \underline{\color{red}inner product} if for scalars $c, d \in \R$,
%
\begin{itemize}

\item[(i)]  $\langle c\vec{u} + d\vec{v},\vec{w}\rangle = c\langle\vec{u},\vec{w}\rangle + d\langle\vec{v},\vec{w}\rangle$, and

$\langle\vec{u},c\vec{v}+d\vec{w}\rangle = c\langle\vec{u},\vec{v}\rangle + d\langle\vec{u},\vec{w}\rangle$.

\item[(ii)]  $\langle \vec{v},\vec{w}\rangle = \langle \vec{w},\vec{v}\rangle$

\item[(iii)]  $\langle\vec{v},\vec{v}\rangle > 0$ whenever $\vec{v} \neq \vec{0}$, and
$\langle\vec{v},\vec{v}\rangle = 0$ only if $\vec{v} = \vec{0}$.

\end{itemize}
\end{definition}

{\color{blue}A dot product is a type of inner product, and that will be the only inner product we will use for now.}

\begin{definition}
If $\langle\vec{v},\vec{w}\rangle = 0$, $\vec{v}$ and $\vec{w}$ are said to be \underline{\color{red}orthogonal}.
\end{definition}

{\color{blue}For vectors in $\R^n$ this corresponds to $\vec{v}\cdot\vec{w}$ and $\vec{v}$ and $\vec{w}$ are said to be perpendicular}.

\bigskip

\underline{Orhtogonality}

We notice that right angles are the most important angles in linear algebra.  Recall that the four fundamental
subspaces meet at right angles.  Also our standard bases are produced using orthogonal vectors, in fact they
are even better; they are unit vectors, which is referred to as \underline{normal}.  So, $\hat{i}, \hat{j}, \hat{k}$
are \underline{orthonormal}.

\begin{itemize}

\item[Ex:  ]  The following vectors are orthogonal.
%
\begin{equation*}
\begin{pmatrix}
-1\\
2
\end{pmatrix}\cdot \begin{pmatrix}
4\\
2
\end{pmatrix} = 0.
\end{equation*}

\item[Ex:  ]  Same as above
%
\begin{equation*}
\begin{pmatrix}
2\\
2\\
-1
\end{pmatrix}\cdot \begin{pmatrix}
-1\\
2\\
2
\end{pmatrix} = 0.
\end{equation*}

\end{itemize}

\begin{thm}
If nonzero vectors $v_1,\ldots, v_k$ are mutually orthogonal (every vector is perpendicular to every
other vector), then those vectors are linearly independent.
\end{thm}

An example of mutually orthogonal vectors are the standard basis vectors: $\hat{i}$, $\hat{j}$, $\hat{k}$ in $\R^3$.  They are clearly linearly independent.  Notice that they are also normal.

There are times when orthogonal will not mean exactly right angles, such as when we look at functions:
%
\begin{equation*}
\begin{pmatrix}
\cos\theta\\
\sin\theta
\end{pmatrix}\cdot \begin{pmatrix}
-\sin\theta\\
\cos\theta
\end{pmatrix}
\end{equation*}
%
Notice that these vectors are also normal since $\cos^2\theta + \sin^2\theta = 1$.  However, these vector functions are not in euclidean space, but rather in function space, so our concept of perpendicular does not hold for these.

\begin{definition}
A basis is an \underline{\color{red}orthonomral basis} if it consists of mutually orthogonal unit vectors; i.e., perpendicular
vectors of length one.
\end{definition}

Also, we notice that we can only have certain combinations of orthogonal vectors in a finite subspace.
Take $\R^3$ for example.  We can only have two lines or a line and a plane, but we cannot have two planes
orthogonal to each other.

\begin{itemize}

\item[Ex:  ]  In $\R^4$ suppose $V$ is the plane spanned by the vectors $v_1 = (1,0,0,0)$ and $v_2 = (1,1,0,0)$.  If $W$ is the line spanned by $w = (0,0,4,5)$, then $w$ is orthogonal to both $v_1$ and $v_2$.  So, the subspaces $W$ and $V$ are mutually orthogonal.

\end{itemize}

\begin{thm}[orthogonality]
The row space $\mathcal{C}(A^T)$ and the nullspace $\mathcal{N}(A)$ are orthogonal to each other, as are
the column space $\mathcal{C}(A)$ and the left nullspace $\mathcal{N}(A^T)$.

\begin{proof}
Notice that $Ax = 0$ for $\mathcal{N}(A)$, but the nonzero rows of $A$ make up $\mathcal{C}(A^T)$.
So, each $x \in \mathcal{N}(A)$ is orthogonal to each row of $A$.
\end{proof}
\end{thm}

\end{document}