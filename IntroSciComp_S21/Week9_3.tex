\documentclass[reqno]{amsart}


\pagestyle{empty}

\usepackage{graphicx}
\usepackage[margin = 1cm]{geometry}
\usepackage{color}
\usepackage{cancel}
\usepackage{multirow}
\usepackage{framed}
\usepackage{algorithm}
\usepackage{algorithmic}
\usepackage{amssymb}
\usepackage{stackengine}
\usepackage{wrapfig}


\newtheorem{thm}{Theorem}
\newtheorem{cor}{Corollary}
\theoremstyle{definition}
\newtheorem{definition}{Definition}

\newenvironment{handwave}{%
  \renewcommand{\proofname}{Handwavey proof}\proof}{\endproof}
  %\renewcommand{\qedsymbol}{$\blacksquare$}

\begin{document}
\begin{flushleft}
{\sc \Large AMATH 301 Rahman} \hfill Week 9 Theory Part 2
\bigskip
\end{flushleft}

\newcommand{\R}{\mathbb{R}}
\newcommand{\N}{\mathbb{N}}
\newcommand{\Z}{\mathbb{Z}}
\newcommand{\Q}{\mathbb{Q}}
\renewcommand{\CancelColor}{\color{red}}
\newcommand{\?}{\stackrel{?}{=}}
\renewcommand{\varphi}{\phi}
\newcommand{\card}{\text{Card}}
\newcommand{\bigzero}{\text{\Huge 0}}
\newcommand{\curvearrowdown}{{\color{red}\rotatebox{90}{$\curvearrowleft$}}}
\newcommand{\curvearrowup}{{\color{red}\rotatebox{90}{$\curvearrowright$}}}



\section*{Week 9 Part 3:  Boundary Value Problems}

We are used to initial value problems where we are given initial data.  What if we are given boundary
data instead?  There are many applications where things are happening for a long period of time and
we don't know what happened in the beginning, but we do know something about the boundary.
The usual problems are solved in a similar fashion to Initial Value Problems.  We do however have
a bit more theory.

\begin{definition}
The boundary values (for a second order ODE) $y(a)$, $y(b)$, $y'(a)$, and/or $y'(b)$
are said to be \underline{homogeneous} if any two of the above boundary data are zero.
\end{definition}

We also have eigenvalue problems for BVPs.  Recall that for matrices the eigenvalue problems
were of the form $Ax = \lambda x$, where we solve for the ``eigenvalue'', $\lambda$.
For BVPs of a second order ODE, we consider our linear operator to be $L = d^2/dx^2$
(for matrices the linear operator is the matrix $A$).  So we wish to solve the problem
$Ly = \lambda y$; i.e. $y'' + \lambda y = 0$.  Here the $y_n's$ corresponding to $\lambda_n's$
are called eigenfunctions (similar to eigenvectors in the matrix case).  We notice that eigenvalue
problems are only for homogeneous boundary data.

\begin{definition}
The boundary value problem
%
\begin{equation}
y'' + \lambda y = 0; \qquad \text{(with homogeneous boundary conditions)},
\end{equation}
%
is called an \underline{eigenvalue problem}.  And the nontrivial (i.e. $y_n \neq 0$) solutions
$y_n$ corresponding to $\lambda_n$ are the \underline{eigenfunctions} of the corresponding
\underline{eigenvalues}.
\end{definition}

\bigskip
\bigskip
\bigskip
\bigskip


Now lets do some boundary value problems,
%
\begin{enumerate}
\setlength\itemsep{2em}

\item[Ex:  ]  $y'' + y = 0$; $y'(0) = 1$, $y(L) = 0$.

\textbf{Solution:  }  The characteristic polynomial gives us
%
\begin{equation*}
r^2 + 1 = 0 \Rightarrow r = \pm i \Rightarrow y = A\cos t + B\sin t \Rightarrow y' = -A\sin t + B\cos t.
\end{equation*}
%
Then our first boundary condition gives $y'(0) = B = 1$, and
%
\begin{equation*}
y(L) = A\cos L + \sin L = 0 \Rightarrow A = - \tan L;\; L \neq (2k+1)\frac{\pi}{2},\, k = 0,\,\pm 1,\,\pm 2,\, \ldots
\end{equation*}
%
However, if $\cos L = 0$, $\sin L = 0$, but this is clearly false because $\sin x \neq 0$ when $\cos x = 0$
and vice-versa, so the BVP has no solution if $L = (2k+1)\frac{\pi}{2}$.

\item[Ex:  ]  $y'' + \lambda y = 0$; $y'(0) = y'(\pi) = 0$.

\textbf{Solution:  }

\textbf{(i)} If $\lambda > 0$, let $\lambda = \mu^2$.  Then
%
\begin{equation*}
r = \pm i\mu \Rightarrow y = A\cos\mu t + B\sin\mu t \Rightarrow y' = -A\mu\sin\mu t + B\mu\cos\mu t
\end{equation*}
%
From the first boundary condition we get $y'(0) = B\mu = 0 \Rightarrow B = 0$ because $\lambda > 0$.
From the second B.C. we get $y'(\pi) = -A\mu\sin\mu\pi = 0$.  Since we don't want trivial solutions if we
can avoid them we can't have $A = 0$, so we require $\sin\mu\pi = 0$ then $\mu = n\pi$ where
$n = 1,\, 2,\, \ldots$, so our eigenfunctions for the corresponding eigenvalues are
%
\begin{equation*}
y_n = \cos n\pi t;\; \lambda_n = n^2,\, n = 1,\, 2,\, \ldots
\end{equation*}
%
\textbf{(ii)}  If $\lambda < 0$, let $\lambda = - \mu^2$.  Then
%
\begin{equation*}
r = \pm \mu \Rightarrow y = c_1e^{\mu t} + c_2e^{-\mu t} = A\cosh\mu t + B\sinh\mu t
\Rightarrow y' = A\sinh\mu t + B\cosh\mu t.
\end{equation*}
%
The B.C.'s give $y'(0) = B\mu = 0 \Rightarrow B = 0$ and $y'(\pi) = A\sinh\mu\pi = 0$, but
$\sinh$ is only zero at zero and $\mu \neq 0$ since $\lambda < 0$, so we have $A = 0$.
Then $y \equiv 0$, so unfortunately we get a trivial solution.
%
\textbf{(iii)}  If $\lambda = 0$, $y = c_1x + c_0 \Rightarrow y' = c_1$, then applying the B.C.'s
give $y'(0) = c_1 = 0$ and $y'(\pi) = 0$ automatically.  Then our eigenvalue and eigenfunction
are
%
\begin{equation*}
y_0 = 1,\, \lambda_0 = 0.
\end{equation*}
%
Notice I left out the constants.  It is up to you if you want to include it or not.

\textbf{Complete Solution:  }

\begin{equation*}
y = c_0 + \sum_{n=1}^\infty A_n \cos n\pi t
\end{equation*}

\item[Ex:  ]  $y'' + \lambda y = 0$; $y'(0) = y(L) = 0$

\textbf{Solution:  }

\textbf{(i)}  If $\lambda > 0$, let $\lambda = \mu^2$, then
%
\begin{equation*}
y = A\cos\mu t + B\sin\mu t \Rightarrow y' = -A\mu\sin\mu t + B\mu\cos\mu t.
\end{equation*}
%
Notice how we have the same exact general solution!  You do enough of these problems
and you can go straight to the solution and it's derivative without having to do the characteristic
polynomial.  Now, from the B.C.'s we get $y'(0) = B\mu = 0 \Rightarrow B = 0$ and
$y(L) = A\cos\mu L = 0$.  So we require $\mu = (2n-1)\pi/2L$ where $n = 1,\, 2,\, 3,\, \ldots$,
then our eigenvalues and eigenfunctions are
%
\begin{equation*}
y_n = A_n\cos\left((2n-1)\frac{\pi}{2}t\right);\; \lambda_n = (2n-1)^2\frac{\pi^2}{4},\, n = 1,\, 2,\, 3,\, \ldots
\end{equation*}
%
\textbf{(ii)}  If $\lambda < 0$, let $\lambda = -\mu^2$, then
%
\begin{equation*}
y = A\cosh\mu t + B\sinh\mu t \Rightarrow y' = A\mu\sinh\mu t + B\mu\cosh\mu t.
\end{equation*}
%
From the B.C.'s we get $y'(0) = B\mu = 0 \Rightarrow B = 0$ and $y(L) = A\cosh\mu L = 0 \Rightarrow A = 0$,
again it's the trivial solution $y \equiv 0$.\\
%
\textbf{(iii)}  If $\lambda = 0$, $y = c_1x + c_0 \Rightarrow y' = c_1$, from the
B.C.'s we get $y'(0) = c_1 = 0$ and $y(L) = c_0 = 0$, so again we have the trivial
solution $y \equiv 0$.

\textbf{Complete Solution:  }
%
\begin{equation*}
y = \sum_{n=1}^\infty A_n\cos\left((2n-1)\frac{\pi}{2}t\right)
\end{equation*}

\end{enumerate}


\end{document}