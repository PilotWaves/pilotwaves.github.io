\documentclass[reqno]{amsart}


\pagestyle{empty}

\usepackage{graphicx}
\usepackage[margin = 1cm]{geometry}
\usepackage{color}
\usepackage{cancel}
\usepackage{multirow}
\usepackage{framed}
\usepackage{algorithm}
\usepackage{algorithmic}
\usepackage{amssymb}
\usepackage{stackengine}


\newtheorem{thm}{Theorem}
\newtheorem{cor}{Corollary}
\theoremstyle{definition}
\newtheorem{definition}{Definition}

\newenvironment{handwave}{%
  \renewcommand{\proofname}{Handwavey proof}\proof}{\endproof}
  %\renewcommand{\qedsymbol}{$\blacksquare$}

\begin{document}
\begin{flushleft}
{\sc \Large AMATH 301 Rahman} \hfill Week 8 Theory Part 3
\bigskip
\end{flushleft}

\newcommand{\R}{\mathbb{R}}
\newcommand{\N}{\mathbb{N}}
\newcommand{\Z}{\mathbb{Z}}
\newcommand{\Q}{\mathbb{Q}}
\renewcommand{\CancelColor}{\color{red}}
\newcommand{\?}{\stackrel{?}{=}}
\renewcommand{\varphi}{\phi}
\newcommand{\card}{\text{Card}}
\newcommand{\bigzero}{\text{\Huge 0}}
\newcommand{\curvearrowdown}{{\color{red}\rotatebox{90}{$\curvearrowleft$}}}
\newcommand{\curvearrowup}{{\color{red}\rotatebox{90}{$\curvearrowright$}}}



\section*{Week 8 Part 3:  Euler's Method}

Numerical solutions to ODEs are all about approximating a derivative and using that to approximate the solution.  We already did that in Week 7, so lets use those approximations.

Consider, $y' = f(t,y)$.  Recall
%
\begin{equation*}
y' = \frac{y(t+h) - y(t)}{h}
\end{equation*}
%
Let $t = t_n$ and $t+h = t_{n+1}$.  Further, we write $y(t_n) = y_n$ and $y(t_{n+1}) = y_{n+1}$.  Then
%
\begin{equation*}
\frac{y_{n+1} - y_n}{h} \approx f(t_n,y_n),
\end{equation*}
%
and therefore
%
\begin{equation}
y_{n+1} = y_n + hf(t_n,y_n);\qquad y_0 = y(t_0)
\end{equation}
%
{\color{red} This is called the \underline{Forward Euler} method because we use the forward difference}.

We can also do a backward approximation.
%
\begin{equation*}
y' = \frac{y(t) - y(t-h)}{h}
\end{equation*}
%
Let $t = t_{n+1}$ and $t+h = t_n$.  Further, we write $y(t_n) = y_n$ and $y(t_{n+1}) = y_{n+1}$.  Then
%
\begin{equation*}
\frac{y_{n+1} - y_n}{h} \approx f(t_{n+1},y_{n+1}),
\end{equation*}
%
and therefore
%
\begin{equation}
y_{n+1} = y_n + hf(t_{n+1},y_{n+1});\qquad y_0 = y(t_0)
\end{equation}
%
{\color{red} This is called the \underline{Backward Euler} method because we use the forward difference}.
{\color{blue} Notice here we have to solve for $y_{n+1}$ since it is implicitly in $f()$}.


\bigskip
\bigskip
\bigskip
\bigskip

If you didn't get a chance to take a look a the Week 7 lectures, here is an alternate, but equivalent derivation of the Forward Euler's method:
%
\begin{equation*}
y'(t) = f(t,y) \Rightarrow f(t,y) \approx \frac{\Delta y}{\Delta t} = \frac{y-y_0}{t-t_0}.
\end{equation*}
%
Now lets evaluate $f$ at $t_1,y_1$, then we get,
%
\begin{equation*}
f(t_1,y_1) \approx \frac{y_1 - y_0}{t_1 - t_0} \Rightarrow y_1 - y_0 \approx (t_1 - t_0)f(t_0,y_0)
\Rightarrow y_1 \approx y_0 + (t_1-t_0)f(t_0,y_0).
\end{equation*}
%
Look at that!  We just developed a formula to approximate $y$ at $t_1$
by using the information we had for the system at $t_0$.  If we can approximate
the data at $t_1$ by using the previous time (i.e. $t_0$), why can't we do this
for any time?  That is we can approximate $y$ at $t_{n+1}$ via the formula,
$y_{n+1} \approx y_n + \Delta tf(t_n,y_n)$.  The standard way to write
this however is with, $h = \Delta t$, basically a renaming and we
usually use $y_0 = y(t_0)$, i.e. the initial condition, and we also drop the
$\approx$ and us $=$.  So our general formula is,
%
\begin{equation}
y_{n+1} = y_n + hf(t_n,y_n);\; y_0 = y(t_0).
\end{equation}
%

\pagebreak

When debugging your codes use the following example, and make sure your values
are close to mine.  Your values might be ever so slightly off, but not more
than say \verb|1e-8|.

\begin{enumerate}

\item  $f(t,y) = 3 + t - y$, which gives us the equation $y_{n+1} = y_n + h\cdot(3 + t_n - y_n)$
where $y_0 = 1$.

\begin{enumerate}

\item  Here we have $h = 0.1$, so we have the following t's.  We get them just by
starting at $t_0$ and incrementing.  $t_0 = 0$, $t_1 = 0.1$, $t_2 = 0.2$, $t_3 = 0.3$,
$t_4 = 0.4$.  Then we have, $y_1 = y_0 + h\cdot(3+t_0-y_0) = 1 + (0.1)(3+0-1) = 1.2$,
$y_2 = y_1 + h\cdot(3+t_1-y_1) = 1.39$, $y_3 = y_2 + h\cdot(3+t_2-y_2) = 1.571$,
and $y_4 = y_3 + h\cdot(3+t_3-y_3) = 1.7439$.  Lets put this in a table to make it
look pretty,

\begin{tabular}{|c||c|c|c|c|c|}
\hline
$n$ & $0$ & $1$ & $2$ & $3$ & $4$\\
\hline
$t_n$ & $0$ & $0.1$ & $0.2$ & $0.3$ & $0.4$\\
\hline
$y_n$ & $1$ & $1.2$ & $1.39$ & $1.571$ & $1.7439$\\
\hline
\end{tabular}

\item  Hopefully part a gave you a good idea of how we do these problems, so I'll just give
the table of values I received when running my code on matlab (remember $h = 0.05$):

\begin{tabular}{|c||c|c|c|c|c|c|c|c|c|}
\hline
$n$ & $0$ & $1$ & $2$ & $3$ & $4$ & $5$ & $6$ & $7$ & $8$\\
\hline
$t_n$ & $0$ & $0.05$ & $0.1$ & $0.15$ & $0.2$ & $0.25$ & $0.3$ & $0.35$ & $0.4$\\
\hline
$y_n$ & $1$ & $1.1$ & $1.1975$ & $1.2926$ & $1.3855$ & $1.4762$ & $1.5649$
& $1.6517$ & $1.7366$\\
\hline
\end{tabular}

\item  Here $h = 0.025$,

\begin{tabular}{|c||c|c|c|c|c|c|c|c|c|}
\hline
$n$ & $0$ & $1$ & $2$ & $3$ & $4$ & $5$ & $6$ & $7$ & $8$\\
\hline
$t$ & $0$ & $0.025$ & $0.05$ & $0.075$ & $0.1$ & $0.125$ & $0.15$ & $0.175$ & $0.2$\\
\hline
$y$ & $1$ & $1.05$ & $1.0994$ & $1.1481$ & $1.1963$ & $1.2439$ & $1.2909$ & $1.3374$ & $1.3833$\\
\hline
\end{tabular}

\begin{tabular}{|c||c|c|c|c|c|c|c|c|}
\hline
$n$ & $9$ & $10$ & $11$ & $12$ & $13$ & $14$ & $15$ & $16$ \\
\hline
$t$ & $0.225$ & $0.25$ & $0.275$ & $0.3$ & $0.325$ & $0.35$ & $0.375$ & $0.4$\\
\hline
$y$ & $1.4288$ & $1.4737$ & $1.5181$ & $1.562$ & $1.6055$ & $1.6484$ & $1.6910$ & $1.7331$\\
\hline
\end{tabular}

\item  Next we solve the equation via integrating factors to get $y = 2+t-e^{-t}$,
and calculating the points gives us the following comparison,

\begin{tabular}{|c|c||c|c|c|c|}
\hline
$h$ & $t=$ & $0.1$ & $0.2$ & $0.3$ & $0.4$\\
\hline
$0.1$ & $y(t)=$ & $1.2$ & $1.39$ & $1.571$ & $1.7439$\\
\hline
$0.05$ & $y(t)=$ & $1.1975$ & $1.3855$ & $1.5649$ & $1.7366$\\
\hline
$0.025$ & $y(t)=$ & $1.1963$ & $1.3833$ & $1.562$ & $1.7331$\\
\hline
Exact & $y(t)=$ & $1.19516$ & $1.38127$ & $1.55918$ & $1.72968$\\
\hline
\end{tabular}
\end{enumerate}
\end{enumerate}

\end{document}