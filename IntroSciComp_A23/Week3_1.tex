\documentclass[reqno]{amsart}


\pagestyle{empty}

\usepackage{graphicx}
\usepackage[margin = 1cm]{geometry}
\usepackage{color}
\usepackage{cancel}
\usepackage{multirow}
\usepackage{framed}
\usepackage{algorithm}
\usepackage{algorithmic}

\newtheorem{thm}{Theorem}
\newtheorem{cor}{Corollary}
\theoremstyle{definition}
\newtheorem{definition}{Definition}

\begin{document}
\begin{flushleft}
{\sc \Large AMATH 301 Rahman} \hfill Week 3 Theory Part 1
\bigskip
\end{flushleft}

\newcommand{\R}{\mathbb{R}}
\newcommand{\N}{\mathbb{N}}
\newcommand{\Z}{\mathbb{Z}}
\newcommand{\Q}{\mathbb{Q}}
\renewcommand{\CancelColor}{\color{red}}
\newcommand{\?}{\stackrel{?}{=}}
\renewcommand{\varphi}{\phi}
\newcommand{\card}{\text{Card}}
\newcommand{\bigzero}{\text{\Huge 0}}



\section*{Week 3 Part 1:  Linear Systems}

Linear equations are equations that create a line.  One main goal of linear algebra is to find where
lines intersect in many dimensions (e.g., millions of dimensions).  Lets look at some examples.

\begin{itemize}

\item[Ex:  ]  $x + y = 1$ and $x - y = 0$.

 \textbf{Solution}: $x = y \Rightarrow 2x = 1 \Rightarrow \fbox{x = y = 1/2}$.

\item[Ex:  ]  $x + 2y = 3$ and $4x + 5y = 6$. 

\textbf{Solution}: $x = 3 - 2y \Rightarrow 4x + 5y = 12 - 8y + 5y = 12 - 3y = 6$,
so $\fbox{y = 2} \Rightarrow \fbox{x = -1}$.

\end{itemize}

We learned to solve linear equations in this manner in high school.  Perhaps we even learned how to
isolate terms through multiplying and adding equations.  However, these methods would not work on
a computer, which is what we need for say a million equations with a million unknowns.  Lets think about
how a computer might be able to solve these problems.

\begin{itemize}

\item[Ex:  ]  $x + 2y = 3$ and $4x + 5y = 6$. 

\textbf{Solution}:
\begin{equation*}
\begin{split}
x + 2y = 3\\
(4x + 5y = 6) - {\color{red}4}{\color{blue}(x + 2y = 3)}
\end{split} \Rightarrow 
\begin{split}
x + 2y = 3\\
0 - 3y = -6
\end{split} \Rightarrow
\fbox{y = 2} \Rightarrow x+4 = 3 \Rightarrow \fbox{x = -1}
\end{equation*}

\end{itemize}

This is called \underline{Gaussian Elimination}.

This seems dumb for only two equations, but for a million by million your computer can probably do
this in less than a second.  In order to tell the computer to conduct this operation, you must first
know how to do it yourself.

Sometimes a system of equations can have \textbf{no solution} or \textbf{infinitely many solutions}.
If it has \textbf{infinitely many solutions} the system is called a \underline{homogeneous system}, and if it has 
\textbf{no solutions} it is called a \underline{singular system}.  However, as we will see later, a matrix
will be called singular if the final system has infinitely many solutions or no solutions.

\begin{itemize}

\item[Ex:  ]  $x + y = 1$, $2x + 2y = 2$.

Notice that an infinite number of $x$ and $y$ combinations (e.g., $x = 1, y = 0$; $x = 0, y = 1$; $x = 1/2, y = 1/2$; etc)

\item[Ex:  ]  $x + 2y = 3$, $4x + 8y = 6$.

Notice that Eq1 - Eq2 gives us $0 = -6$, and therefore does not have a solution.

\end{itemize}





\end{document}