\documentclass[reqno]{amsart}


\pagestyle{empty}

\usepackage{graphicx}
\usepackage[margin = 1cm]{geometry}
\usepackage{color}
\usepackage{cancel}
\usepackage{multirow}
\usepackage{framed}
\usepackage{amssymb}
\usepackage{stackengine}
\usepackage{tikz}

\newtheorem{thm}{Theorem}
\newtheorem{cor}{Corollary}
\theoremstyle{definition}
\newtheorem{definition}{Definition}

\newenvironment{handwave}{%
  \renewcommand{\proofname}{Handwavey proof}\proof}{\endproof}
  %\renewcommand{\qedsymbol}{$\blacksquare$}

\begin{document}
\begin{flushleft}
{\sc \Large AMATH 352 Rahman} \hfill Week 7
\bigskip
\end{flushleft}

\newcommand{\R}{\mathbb{R}}
\newcommand{\N}{\mathbb{N}}
\newcommand{\Z}{\mathbb{Z}}
\newcommand{\Q}{\mathbb{Q}}
\renewcommand{\CancelColor}{\color{red}}
\newcommand{\?}{\stackrel{?}{=}}
\renewcommand{\varphi}{\phi}
\newcommand{\card}{\text{Card}}
\newcommand{\bigzero}{\text{\Huge 0}}
\newcommand{\curvearrowdown}{{\color{red}\rotatebox{90}{$\curvearrowleft$}}}
\newcommand{\curvearrowup}{{\color{red}\rotatebox{90}{$\curvearrowright$}}}

\newcommand*\circled[1]{\color{red}\tikz[baseline=(char.base)]{
            \node[shape=circle,draw,inner sep=2pt] (char) {#1};}}



\section*{Sec. 3.2 Inequalities}

{\color{blue} I am going to do this slightly differently from the book.}

Lets first talk about projections onto a line.

\bigskip

\underline{Projections onto lines}

Please refer to the video for the sketch.

The shortest distance from a point $\vec{b}$ onto a line through $\vec{a}$ is via a line through $\vec{b}$
that is perpendicular to the line through $\vec{a}$.  This will meet the line through $\vec{a}$ at a point
$\vec{p}$.  Then $\vec{p}$ is simply a scaled version of $\vec{a}$, so $\vec{p} = \hat{x}\vec{a}$,
and the perpendicular line is $\vec{b} - \vec{p}$.  Then $\vec{a}^T(b - \hat{x}\vec{a}) = 0$, and solving
for $\hat{x}$ give us $\hat{x} = (\vec{a}^T\vec{b})/(\vec{a}^T\vec{a})$.

\begin{definition}
The projection of the vector $\vec{b}$ onto the line in the direction of $\vec{a}$ is 
%
\begin{equation}
\vec{p} = \hat{x}\vec{a} = \left(\frac{\vec{a}^T\vec{b}}{\vec{a}^T\vec{a}}\right)\vec{a}.
\end{equation}
\end{definition}

We will use projections for least squares solutions to singular systems, and to derive
the Cauchy-Schwartz inequality.  Before we go into that, lets first do some examples with projections.

\begin{itemize}

\item[Ex:  ]  Project $b = (1, 2, 3)$ onto the line through $a = (1, 1, 1)$ to get $\hat{x}$ and $p$.

\textbf{Solution:  }
%
\begin{equation*}
\hat{x} = \frac{a^Tb}{a^Ta} = \frac{6}{3} = 2 \Rightarrow p = \hat{x}a = (2,2,2).
\end{equation*}

\item[Ex:    ]  Consider the vectors
%
\begin{equation*}
\vec{u} = \begin{bmatrix}
5\\
-3\\
1
\end{bmatrix},\qquad \vec{v} = \begin{bmatrix}
1\\
-1\\
0
\end{bmatrix}
\end{equation*}

\begin{enumerate}

\item  For $u$ onto $v$ we do
%
\begin{equation*}
\hat{x} = \frac{v^Tu}{v^Tv} = \frac{8}{2} = 4 \Rightarrow p = \hat{x}v = (4, -4, 0)
\end{equation*}

\item  For $v$ onto $u$ we do
%
\begin{equation*}
\hat{x} = \frac{u^Tv}{u^Tu} = \frac{8}{35} \Rightarrow p = \hat{x}u = \left(\frac{8}{7},-\frac{24}{35},\frac{8}{35}\right)
\end{equation*}

\end{enumerate}

\end{itemize}

\bigskip

\underline{The Cauchy-Schwartz inequality}

As mentioned before, this leads us to perhaps the most important inequality in mathematics.  Lets derive it here.

Recall that the error vector $e = b-p \Rightarrow \|e\| = \|b-p\|$, then writing out the equation for the projection vector and squaring gives us
%
\begin{equation*}
\left\|b - \frac{a^Tb}{a^Ta}a\right\|^2 = \left(b - \frac{a^Tb}{a^Ta}a\right)^T\left(b - \frac{a^Tb}{a^Ta}a\right) = b^Tb - 2\frac{(a^Tb)^2}{a^Ta} + \left(\frac{a^Tb}{a^Ta}\right)^2a^Ta = \frac{(b^Tb)(a^Ta) - (a^Tb)^2}{a^Ta} \geq 0
\end{equation*}
%
since we squared the left hand side.  Since the denominator is finite,
%
\begin{equation*}
(b^Tb)(a^Ta) - (a^Tb)^2 \geq 0 \Rightarrow (a^Tb)^2 \leq \|a\|^2\|b\|^2.
\end{equation*}
%

Another way we can derive this inequality is through the law of cosines.
%
\begin{equation*}
\left\vert\frac{a^Tb}{\|a\|\|b\|}\right\vert = |\cos\theta| \leq 1 \Rightarrow |a^Tb| \leq \|a\|\|b\|.
\end{equation*}


\begin{thm}
All inner products $\langle a,b\rangle$ satisfy the \underline{\color{red}Cauchy-Schwartz inequality}
%
\begin{equation}
\left\vert a^Tb\right\vert = \|a\|\|b\|.
\end{equation}
\end{thm}




\end{document}