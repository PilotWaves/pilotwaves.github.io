\documentclass[reqno]{amsart}


\pagestyle{empty}

\usepackage{graphicx}
\usepackage[margin = 1cm]{geometry}
\usepackage{color}
\usepackage{cancel}
\usepackage{multirow}
\usepackage{framed}
\usepackage{amssymb}
\usepackage{stackengine}
\usepackage{tikz}

\newtheorem{thm}{Theorem}
\newtheorem{cor}{Corollary}
\theoremstyle{definition}
\newtheorem{definition}{Definition}

\newenvironment{handwave}{%
  \renewcommand{\proofname}{Handwavey proof}\proof}{\endproof}
  %\renewcommand{\qedsymbol}{$\blacksquare$}

\begin{document}
\begin{flushleft}
{\sc \Large AMATH 352 Rahman} \hfill Week 8
\bigskip
\end{flushleft}

\newcommand{\R}{\mathbb{R}}
\newcommand{\N}{\mathbb{N}}
\newcommand{\Z}{\mathbb{Z}}
\newcommand{\Q}{\mathbb{Q}}
\renewcommand{\CancelColor}{\color{red}}
\newcommand{\?}{\stackrel{?}{=}}
\renewcommand{\varphi}{\phi}
\newcommand{\card}{\text{Card}}
\newcommand{\bigzero}{\text{\Huge 0}}
\newcommand{\curvearrowdown}{{\color{red}\rotatebox{90}{$\curvearrowleft$}}}
\newcommand{\curvearrowup}{{\color{red}\rotatebox{90}{$\curvearrowright$}}}

\newcommand*\circled[1]{\color{red}\tikz[baseline=(char.base)]{
            \node[shape=circle,draw,inner sep=2pt] (char) {#1};}}



\section*{5.3 Gram-Schmidt}

By now we are used to finding bases, but recall that orthogonal, or even better, orthonormal bases are preferred.

\begin{definition}
The vectors $q_1,\ldots,q_n$ are \underline{orthonormal} if
%
\begin{equation}
q_i^Tq_j = \begin{cases}
0 & \text{  if $i \neq j$ \qquad {\color{blue}(giving orthogonality)}},\\
1 & \text{  if $i = j$ \qquad {\color{blue}(giving the normalization)}};
\end{cases}
\end{equation}
\end{definition}

We can also create matrices out of these bases.  Notice that the standard basis for an Euclidean space
is in the columns of the identity matrix.  However, if we want a generic orthonormal basis we need to
apply the \underline{Gram-Schmidt orthogonalization} procedure.

\begin{thm}
If $Q$ (square or rectangular) has orthonormal columns, then $Q^TQ = 1$.
\end{thm}

\begin{definition}
An \underline{orthogonal matrix} is a square matrix with orthonormal columns.
\end{definition}

\begin{thm}
For orthogonal matrices, the transpose is the inverse.
\end{thm}

\begin{itemize}

\item[Ex:  ]  Consider
%
\begin{equation*}
Q = \begin{pmatrix}
\cos\theta & -\sin\theta\\
\sin\theta & \cos\theta
\end{pmatrix} \Rightarrow Q^T = Q^{-1} = \begin{pmatrix}
\cos\theta & \sin\theta\\
-\sin\theta & \cos\theta
\end{pmatrix}
\end{equation*}
%
which we can verify by multiplying.

\item[Ex:  ]  Any permutation matrix $P$ (consisting of only row exchanges) is an orthogonal matrix.
The criteria of orthonormal columns and square are trivially satisfied.  Then we check $P^{-1} = P^T$
by checking $PP^T = I$.

\end{itemize}

\begin{thm}
Multiplication by any $Q$ preserves lengths:  $||Qx|| = ||x||$ for all $x$.

It also preserves inner products and angles:  $(Qx)^T(Qy) = x^TQ^TQy = x^Ty$.
\end{thm}

Consider $Qx = b$ where $q_i$ are the columns of $Q$.  Then we can write
%
\begin{equation*}
b = x_1q_1 + x_2q_2 + \cdots + x_iq_i + \cdots + x_{n-1}q_{n-1} + x_nq_n
\end{equation*}
%
If we multiply both sides by $q_i^T$ we get
%
\begin{equation*}
q_i^T = 0 + \cdots + x_iq_i^Tq_i + \cdots + 0 = x_i \Rightarrow x = Q^Tb.
\end{equation*}
%
So if your $A$ is an orthogonal matrix, you don't have to do Gaussian Elimination.

\underline{The Gram-Schmidt Process}

Suppose you are given three independent vectors $\vec{a}, \vec{b}, \vec{c}$.  If they are orthonormal
we can project a vector $\vec{v}$ onto $\vec{a}$ by doing $(\vec{a}^T\vec{v})\vec{a}$.  To project
onto the $\vec{a}-\vec{b}$ plane we do $(\vec{a}^T\vec{v})a + (\vec{b}^T\vec{v})b$, etc.

Process:  We are given $\vec{a}, \vec{b}, \vec{c}$ and we want $\vec{q}_1, \vec{q}_2, \vec{q}_3$.
No problem with $q_1$; i.e., $q_1 = a/||a||$ {\color{red}(we don't have to change its direction, just
normalize.)}  The problem begins with $q_2$, which has to be orthogonal to $q_1$.  If the vector $b$
has any component in the direction of $q_1$ {\color{red}(i.e., direction of $a$)} it has to be
subtracted:  $B = b - (q_1^Tb)q_1$, then $q_2 = B/||B||$, and this continues for $q_3$:
$C = c - (q_1^Tc)q_1 - (q_2^Tc)q_2$, then $q_3 = C/||C||$, so on and so forth.

\begin{itemize}

\item[Ex:  ]  $a = (1, 0, 1)$, $b = (1, 0, 0)$, and $c = (2, 1, 0)$ for $A = [a\qquad b\qquad c]$.

\textbf{Solution:  }

\begin{enumerate}

\item[Step 1:  ]  Make the first vector into a unit vector:  $\boxed{q_1 = a/\sqrt{2} = (1/\sqrt{2}, 0, 1/\sqrt{2})}$.

\item[Step 2a:  ]  Subtract from the second vector its component in the direction of the first:
$\boxed{B = b - (q_1^Tb)q_1 = (1/2, 0, -1/2)}$.

\item[Step 2b:  ]  Divide $B$ by its magnitude:  $\boxed{q_2 = B/||B|| = (1/\sqrt{2}, 0, -1/\sqrt{2})}$.

\item[Step 3a:  ]  Subtract from the third vector its component in the firs and second directions:
$\boxed{C = c - (q_1^Tc)q_1 - (q_2^Tc)q_2 = (0, 1, 0)}$.

\item[Step 3b:  ]  We normalize $C$, but $C$ is already a unit vector so $\boxed{q_3 = (0, 1, 0)}$

\end{enumerate}

Then we can write $Q$ as the matrix
%
\begin{equation*}
Q = \begin{pmatrix}
| & | & |\\
q_1 & q_2 & q_3\\
| & | & |
\end{pmatrix} = \begin{pmatrix}
1/\sqrt{2} & 1/\sqrt{2} & 0\\
0 & 0 & 1\\
1/\sqrt{2} & -1/\sqrt{2} & 0
\end{pmatrix}
\end{equation*}

\end{itemize}

From the matrix $Q$ we can get a $A = QR$ factorization.  This means that $A = QR \Rightarrow Q^TA = R$, then
%
\begin{equation}
R = \begin{pmatrix}
--- & q_1^T & ---\\
--- & q_2^T & ---\\
--- & q_3^T & ---
\end{pmatrix} \begin{pmatrix}
| & | & |\\
a & b & c\\
| & | & |
\end{pmatrix} = \begin{pmatrix}
q_1^Ta & q_1^Tb & q_1^Tc\\
0 & q_2^Tb & q_2^Tc\\
0 & 0 & q_3^Tc
\end{pmatrix}
\end{equation}

\pagebreak

Now lets do a bunch of examples.  I skip some of the computation steps in the notes,
but they are done in detail in the lectures.

\begin{enumerate}

\item[Ex:  ]  Are the vectors orthonormal?
%
\begin{equation*}
\left\lbrace\begin{bmatrix}
2\\
-4
\end{bmatrix}, \begin{bmatrix}
2\\
1
\end{bmatrix}\right\rbrace
\end{equation*}

They are orthogonal but not normal, so they are not orthonormal.

\item[Ex:  ]  Are the vectors orthonormal?
%
\begin{equation*}
\left\lbrace\begin{bmatrix}
4\\
-1\\
1
\end{bmatrix}, \begin{bmatrix}
-1\\
0\\
4
\end{bmatrix}, \begin{bmatrix}
-4\\
-17\\
-1
\end{bmatrix}\right\rbrace
\end{equation*}

They are orthogonal but not normal, so they are not orthonormal.

\item[Ex:  ]  Orthonormalize the following set of vectors
%
\begin{equation*}
\left\lbrace\begin{bmatrix}
3\\
4
\end{bmatrix}, \begin{bmatrix}
1\\
0
\end{bmatrix}\right\rbrace
\end{equation*}

\textbf{Solution:  }
We can get $\boxed{q_1 = (3,4)/5}$ immediately.  Then
%
\begin{equation*}
B = b - (q_1^Tb)q_1 = (1,0) - \frac{3}{5}(3,4)/5 = \boxed{(16/25, -12/25)}
\Rightarrow q_2 = \frac{(4^2/5^2, -12/5^2)}{\sqrt{(4^4/5^4) + (3^2\cdot 4^2)/5^4}}
= \boxed{(4/5, -3/5)}.
\end{equation*}

\item[Ex:  ]Orthonormalize the following set of vectors
%
\begin{equation*}
\left\lbrace\begin{bmatrix}
0\\
1
\end{bmatrix}, \begin{bmatrix}
2\\
5
\end{bmatrix}\right\rbrace
\end{equation*}

\textbf{Solution:  }
$\boxed{q_1 = (0, 1)}$.  Then
%
\begin{equation*}
B = b - (q_1^Tb)q_1 = (2, 5) - 5(0, 1) = \boxed{(2, 0)} \Rightarrow \boxed{q_2 = (1, 0)}.
\end{equation*}

\item[Ex:  ]  Orthonormalize the following set of vectors
%
\begin{equation*}
\left\lbrace\begin{bmatrix}
2\\
1\\
-2
\end{bmatrix}, \begin{bmatrix}
1\\
2\\
2
\end{bmatrix}, \begin{bmatrix}
2\\
-2\\
1
\end{bmatrix}\right\rbrace
\end{equation*}

\textbf{Solution:  }
The vectors are already orthogonal, so just divide by the magnitude.

\item[Ex:  ]Orthonormalize the following set of vectors
%
\begin{equation*}
\left\lbrace\begin{bmatrix}
0\\
1\\
1
\end{bmatrix}, \begin{bmatrix}
1\\
1\\
0
\end{bmatrix}, \begin{bmatrix}
1\\
0\\
1
\end{bmatrix}\right\rbrace
\end{equation*}

\textbf{Solution:  }
$\boxed{q_1 = (0, 1, 1)/\sqrt{2}}$.  Then
%
\begin{equation*}
B = b - (q_1^Tb)q_1 = (1, 1, 0) - \frac{1}{\sqrt{2}}(0, 1/\sqrt{2}, 1/\sqrt{2})
= (1, 1/2, -1/2) \Rightarrow q_2 = (1, 1/2, -1/2)/\sqrt{3/2} = (\sqrt{2/3}, \sqrt{2/3}/2, -\sqrt{2/3}/2).
\end{equation*}
%
And
%
\begin{align*}
C &= c - (q_1^Tc)q_1 - (q_2^Tc)q_2 = (1, 0, 1) - \frac{1}{\sqrt{2}}(0, 1/\sqrt{2}, 1/\sqrt{2}) - \frac{\sqrt{2/3}}{2}
(\sqrt{2/3}, \sqrt{2/3}/2, -\sqrt{2/3}/2) \\
&= (1, 0, 1) - (0, 1/2, 1/2) - (1/3, 1/6, -1/6) = (2/3, -2/3, 2/3).
\end{align*}
%
So,
%
\begin{equation*}
q_3 = (2/3, -2/3, 2/3)/(2\sqrt{2/3}) = (1/\sqrt{3}, -1/\sqrt{3}, 1/\sqrt{3}).
\end{equation*}

\end{enumerate}




\end{document}