%&latex
\documentclass[11pt]{article}
\usepackage{fullpage}

% Title Page
\title{}
\usepackage{amsmath}
\usepackage{amssymb}
\usepackage{amssymb}
\usepackage{graphicx}
\usepackage{varioref}
\usepackage[normalem]{ulem}
\usepackage{hyperref}

\begin{document}

\begin{center} AMATH 481 Autumn 2020 \\
\textbf{Final Project}\\[.3in]
\end{center}

\noindent {\bf Due at 11:59:59pm on Wednesday December 16, 2020}\\

Grading:  the main grading points we will look for are the code, results, and exposition.  Please make sure to have all three in your project for consideration.  Since it is worth 3 Skills there will be partial credit; e.g., if you have a working code that solves your problem and good results in the form of plots, tables, etc., but your exposition is not at an advanced undergraduate level, then you will receive 2 of the 3 skills.\\

Pick the most suitable option for the final project for the course: \\
(1) Write a ``Computational Notebook" that summarizes, explains
and investigates the numerical approaches, i.e., the theory, algorithms and Matlab implementation covered during the course. The aim of this project
is that the
 notebook  will
serve you as a  reference from now on 
to  scientific computing and numerical approaches to differential equations. Organize the notebook
to address   three different major physical problems that we discussed during the
course, e.g., Quantum Harmonic
Oscillator, Vorticity-Streamfunction Equations, Reaction-Diffusion Systems (like a survey paper).
See the lecture notes for additional problems (particularly
at the end).   Be sure to include \textbf{graphical} expositions of the findings and
the results (suitable plots) and explain  them   thoroughly. \\
Make sure to include full derivations and analysis done  in class, e.g. stability analysis, derivation of central difference, etc (at least one for each topic). 
\\\\
(2) Pick a problem from your research/ interest / other course and study it  with the various numerical methods introduced in the course. Submit a \textbf{report} on your investigations. In particular, make sure to explain the
\textbf{problem}, the \textbf{methods} that can apply for solving it numerically and which
approach/es are most applicable.  Include all the \textbf{results}, also the unsuccessful
ones and explain why they did not work and what could be a possible \textbf{extension}
that will make them work better.     Include \textbf{graphical} expositions of the findings  (suitable plots) and  explain them thoroughly.
Discuss how this numerical investigation contributed to the understanding of the problem.\\
Make sure that you have  \textbf{implemented} and explained the code needed for this part by yourselves. Explain your implementation thoroughly and include citations of relevant sources.\\
\\\\
(3) Find a method which was published in a journal and is relevant to the material covered in class, examples: \href{http://www.jstor.org/stable/pdfplus/2157521.pdf?acceptTC=true&jpdConfirm=true}{Split-Step method for NLS}, \href{http://engineering.jhu.edu/fsag/wp-content/uploads/sites/23/2013/10/JCP_revised_WebPost.pdf}{Accelerated Jacobi Method}, etc. \textbf{Describe} (with illustrations and derivations) and \textbf{implement} the method described in the paper. Reproduce some of the figures in the paper.
\\
Discuss the novelty of the method with respect to methods covered in class. 
\\ \\ \\
(4) Solve the Bose-Einstein Condensation problem (Gross-Pitaevskii equations) in \textbf{3D} (the assignment is described in a separate set), submit the solutions to Gradescope and write a \textbf{report} on your \textbf{investigation} and the \textbf{numerical} approach.
\end{document}

